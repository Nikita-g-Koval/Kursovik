% !TeX spellcheck = ru_RU-Russian
\section{Рабочий проект}
\subsection{Классы, используемые при разработке программы}

\subsubsection{Класс Test}

Данный класс необходим для проведения тестирования. Для создания объекта требуется передать название теста и список вопросов. Методы класса описаны в таблице \ref{test_functions:table}. Свойства класса описаны в таблице \ref{test_properties:table}

\begin{xltabular}{\textwidth}{|p{6cm}|X|}
	\caption{Таблица методов класса Test\label{test_functions:table}} \hline
	\centrow Метод & \centrow Описание метода \\ \hline
	\endfirsthead
	\continuecaption{Продолжение таблицы \ref{test_functions:table}}  \hline
	\centrow Метод & \centrow Описание метода \\ \hline
	\finishhead
	print\_current\_question(self) & Возвращает номер вопроса и текст вопроса в виде строки "<№\{Номер вопроса\}. \{Текст вопроса\}">. \\ \hline
	shuffle\_questions(self) & Перемешивает вопросы текущего теста случайным образом, ничего не возвращает. \\ \hline
	shuffle\_answers(self) & Перемешивает ответы текущего вопроса, ничего не возвращает. \\ \hline
	start\_test(self) & Запускает тест, обнуляет счёт и устанавливает id текущего вопроса, ничего не возвращает. \\ \hline
	next\_question(self) & Выполняет инкремент id текущего вопроса и перемешивает его ответы, ничего не возвращает. \\ \hline
	accept\_answers(self, user\_answers: List[Answer]) & Проверяет ответ на базовый вопрос и увеличивает счёт, если ответ правильный. Ничего не возвращает. \\ \hline
	\_increase\_score(self) & Увеличивает счёт, ничего не возвращает. \\ \hline
	calculate\_right\_answers\_percentage(questions\_count, right\_answers\_count) & Статический метод. Расчитывает отношение правильных ответов к количеству вопросов, возвращает процентное соотношение. \\ \hline
	summarise(self) & Возвращает процентное соотношение правильных ответов к общему количеству вопросов.
\end{xltabular}

\begin{xltabular}{\textwidth}{|p{6cm}|X|}
	\caption{Таблица свойств класса Test\label{test_properties:table}} \hline
	\centrow Свойство & \centrow Описание свойства \\ \hline
	\endfirsthead
	\continuecaption{Продолжение таблицы \ref{test_properties:table}}  \hline
	\centrow Свойство & \centrow Описание свойства \\ \hline
	\finishhead
	questions\_count(self) & Возвращает количество вопросов. \\ \hline
	get\_current\_question(self) & Возвращает текущий вопрос. \\ \hline
	get\_current\_answers(self) & Возвращает ответы текущего вопроса. \\ \hline
	is\_finished(self) & Возвращает статус завершения теста.
\end{xltabular}

\subsubsection{Класс FileProvider}

Данный класс статический, и используется другими классами для работы с файловой системой. Методы класса описаны в таблице \ref{fileProvider_functions:table}.

\begin{xltabular}{\textwidth}{|p{6cm}|X|}
	\caption{Таблица методов класса QuestionsStorage\label{fileProvider_functions:table}} \hline
	\centrow Метод & \centrow Описание метода \\ \hline
	\endfirsthead
	\continuecaption{Продолжение таблицы \ref{fileProvider_functions:table}} 
	\centrow Метод & \centrow Описание метода \\ \hline
	\finishhead
	get\_admin\_password() & Обращается к файлу, содержащему пароль администратора, извлекает данные, десерилизует их и возвращает пароль администратора. \\ \hline 
	set\_admin\_password(new\_password) & Принимает новый пароль администратора, обращается к файлу, содержащему пароль администратора, сериализует новый пароль и перезаписывает старый пароль на новый. Если файла не существует, создаёт его. Ничего не возвращает. \\ \hline
	save\_user(user: User) & Принимает объект класса User(пользователь), сериализует данные пользователя и сохраняет в файл, содержащий информацию о пользователях. Если файла не существует, создаёт его. Ничего не возвращает. \\ \hline
	delete\_user(user\_name: str) & Принимает имя пользователя. Находит в файле, содержащем информацию о пользователях, пользователя с переданным именем и удаляет его. Ничего не возвращает. \\ \hline
	get\_users() & Извлекает и десериализует данные о пользователях из соответствующего файла. Возвращает список пользователей. \\ \hline
	save\_test\_result(test\_result: TestResult) & Принимает объект класса TestResult, сериализует его и сохраняет в файл с результатами. Если файла не существует, создаёт его. Ничего не возвращает. \\ \hline
	get\_results() & Извлекает и десериализует результаты прохождения тестов всех пользователей. Возвращает список результатов. \\ \hline
	clear\_test\_results() & Удаляет файл результатов. \\ \hline
	save\_test(test: Test) & Принимает объект класса Test, сериализует его и сохраняет в файл с тестами. Если файла не существует, создаёт его. Ничего не возвращает. \\ \hline
	get\_test(test\_path: str) & Принимает путь к тесту, извлекает данные теста и десериализует их. Возвращает объект класса Test. \\ \hline
	save\_test\_changes(test: Test, test\_path: str) & Принимает объект класса Test и путь для его сохранения. Сериализует данные теста, создаёт файл с именем теста и сохраняет в него данные. Ничего не возвращает. \\ \hline
	get\_test\_names() & Находит все существующие тесты в папке "<Tests">. Возвращает список их названий найденных тестов. \\ \hline
	find\_test\_path(wanted\_test \_name: str) & Принимает имя искомого теста. Находит путь к тесту с данным именем. Возвращает строку - путь к искомому тесту.
\end{xltabular}

\subsubsection{Класс QuestionsStorage}

Данный класс характеризует хранилище вопросов, позволяет добавлять вопрос, удалять вопрос, обновлять список вопросов и сохранять изменения. Для работы с файлами использует методы статического класса FileProvider, который описан выше. Методы класса QuestionsStorage описаны в таблице \ref{questionsStorage_functions:table}.

\begin{xltabular}{\textwidth}{|p{6cm}|X|}
	\caption{Таблица методов класса QuestionsStorage\label{questionsStorage_functions:table}} \hline
	\centrow Метод & \centrow Описание метода \\ \hline
	\endfirsthead
	\continuecaption{Продолжение таблицы \ref{questionsStorage_functions:table}} 
	\centrow Метод & \centrow Описание метода \\ \hline
	\finishhead
	add\_question(self, question) & Добавляет вопрос в хранилище. \\ \hline 
	remove\_question(self, question\_number) & Принимает индекс вопроса. Удаляет вопрос по индексу из хранилища. \\ \hline
	update\_test(self, test\_path: str) & Принимает строку - путь к файлу с вопросами. Обновляет текущий список вопросов из файла по переданному пути. Ничего не возвращает. \\ \hline
	save\_changes(self) & Сохраняет изменения по пути текущего теста. Ничего не возвращает.
\end{xltabular}

\subsubsection{Класс Main}

С помощью класса Main создается графический интерфейс для управления текущими заказами их добавление, удаление, возможность их редактирования, увеличение или уменьшения числа блюд, применение скидок. Методы данного класса представлены в таблице \ref{class:table_class_main}.

\begin{xltabular}{\textwidth}{|p{6cm}|X|}
	\caption{Таблица методов класса Main\label{class:table_class_main}} \hline
	\centrow Метод & \centrow Описание метода \\ \hline
	\endfirsthead
	\continuecaption{Продолжение таблицы \ref{class:table_class_main}} 
	\centrow Метод & \centrow Описание метода \\ \hline
	\finishhead
	load\_existing\_order() & С помощью данного метода производится проверка данных о заказе. Если заказ уже имеется в базе данных, то его данные загружаются в таблицу программы. Если же заказа не существует, то открывается новое пустое окно. \\ \hline 
	load\_order\_data(order\_id) & Данный метод используется для обновления данных о блюдах и стоимости заказов. Данные сверяются с таблицами базы данных с помощью уникального идентификатора заказа order\_id. \\ \hline 
	on\_plus(dish\_name) & Используя уникальный идентификатор dish\_name полученный из строки таблицы программы, метод добавляет единицу к значению количества блюда как в таблице графического интерфейса, так и в базе данных. \\ \hline  
	on\_minus(dish\_name) & Используя уникальный идентификатор dish\_name полученный из строки таблицы программы, метод уменьшает на единицу значение количества блюда как в таблице графического интерфейса, так и в базе данных. \\ \hline 
	on\_delete(dish\_name) & Используя уникальный идентификатор dish\_name полученный из строки таблицы программы, метод удаляет выбранное блюдо как в таблице графического интерфейса, так и в базе данных. \\ \hline 
	on\_quantity(dish\_name) & Используя уникальный идентификатор dish\_name полученный из строки таблицы программы, метод добавляет введённое значению количества блюда как в таблице графического интерфейса, так и в базе данных. 
	
\end{xltabular}




\subsubsection{Класс Payment}


В классе Payment осуществляется инициализация окна оплаты заказа, с возможностью выбора типа оплаты и указания суммы платежа. Класс взаимодействует с базой данных, загружая информацию о заказе и обновляя его статус после подтверждения оплаты.  \ref{class:table_class_payment}.

\begin{xltabular}{\textwidth}{|p{6cm}|X|}
	\caption{Таблица методов класса Payment\label{class:table_class_payment}} \hline
	\centrow Метод & \centrow Описание метода \\ \hline
	\endfirsthead
	\continuecaption{Продолжение таблицы \ref{class:table_class_payment}} 
	\centrow Метод & \centrow Описание метода \\ \hline
	\finishhead
	load\_order\_data(order\_id) & Данный метод используется для загрузки из таблиц БД данных о заказе: блюдах, их количестве, стоимости, скидки, если та была применена и итоговой сумме. \\ \hline
	save\_order\_data() & В данном методе производится проверка внесения оплаты (её количества) по средствам уменьшения суммы остатка - переменной "<balance">. Если значение остатка меньше или равно нулю, тогда статус заказа меняется на "Оплачен" и данные сохраняются в базу данных. В противном же случаие, появится сообщение об ошибке. \\ \hline 
	save\_payment\_data() & Передает данные о всех типах оплаты и суммах в таблицу базы данных.
\end{xltabular}

\subsubsection{Класс ManagerStaff}

Класс ManagerStaff представляет собой графическое окно управления данными о сотрудниках. Функционалом класса является отображение, добавление, редактирования, удаления, поиска и обновления данных, хранящихся в таблицах базы. Реализованные методы класса представлены в таблице  \ref{class:table_class_staff}.

\begin{xltabular}{\textwidth}{|p{6cm}|X|}
	\caption{Таблица методов класса Payment\label{class:table_class_staff}} \hline
	\centrow Метод & \centrow Описание метода \\ \hline
	\endfirsthead
	\continuecaption{Продолжение таблицы \ref{class:table_class_staff}} 
	\centrow Метод & \centrow Описание метода \\ \hline
	\finishhead
	tabel() & Метод используется для создания таблицы, в которой отображаются данные о сотрудниках.\\ \hline
	data\_BD() & Метод применяется для подключения к базе данных и загрузки сведений о сотрудниках их таблиц staff и post. \\ \hline 
	save\_payment\_data() & Передает данные о всех типах оплаты и суммах в таблицу базы данных. \\ \hline 
	add\_data() & Данный метод вызывает функцию create\_staff\_window для открытия окна добавления данных о новом сотруднике как в таблицу программы, так и в таблицу БД.  \\ \hline 
	edit\_data() & Как и в предыдущем методе происходит вызов функции create\_staff\_window  с передачей в нее уже существующих данных о работнике, ранее выделенном путем нажатия на строку с выбранным сотрудником. \\ \hline
	delete\_data() & Метод получает выделенного сотрудника, запрашивает подтверждение удаления сотрудника по фамилии и коду из таблицы базы данных staff. После удаления таблица программы автоматически обновляется. \\ \hline
	window\_search\_data() & Данный метод создает окно для ввода поискового запроса. Поиск производится по имени, фамилии или должности. После нажатия кнопки "<Поиск"> выполняется запрос к базе данных с фильтрацией по введённому тексту. Таблица программы обновляется и выдает только результаты соответствующие поисковому запросу. После выполнения всех вышеперечисленных действий окно поиска закрывается.  \\ \hline
	refresh\_data() & Для обновления данных в таблице программы вызывается метод data\_BD().
	
\end{xltabular}

\subsection{Модульное тестирование разработанного web-сайта}

Модульный тест для класса User из модели данных представлен на рисунке \ref{unitUser:image}.

\begin{figure}[ht]
\begin{lstlisting}[language=Python]
from django.test import TestCase
from .models import *
User = get_user_model()


class ShpoTestCases(TestCase):

    def setUp(self) -> None:
        self.user = User.objects.create(username='testtestovich', password='testtestovich', first_name='Sad', last_name='')

    def test_2(self):

        self.assertEqual(self.user.first_name, 'Sad')
        self.assertEqual(self.user.last_name, 'Cat')
        print((self.user))
        print((self.user.first_name))
        print((self.user.last_name))
\end{lstlisting}  
\caption{Модульный тест класса User}
\label{unitUser:image}
\end{figure}

\subsection{Системное тестирование разработанного web-сайта}

На рисунке  представлена главная страница сайта «Русатом – Аддитивные технологии».
\newpage % при необходимости можно переносить рисунок на новую страницу


На рисунк представлен динамический вывод заголовков, включающий в себя искомые фразы при поиске фраз.


На рисунк представлен ввод данных для публикации новости.

