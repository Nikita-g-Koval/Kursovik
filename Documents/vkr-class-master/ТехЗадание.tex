\section{Техническое задание}
\subsection{Основание для разработки}

Полное наименование системы: «Программно-информационная система для оценки и контроля знаний».
Основанием для разработки программы является приказ ректора ЮЗГУ
от «17» апреля 2025 г. №1828-с «О направлении (допуске) на практику».

\subsection{Цель и назначение разработки}

Программно-информационная система предназначена для контроля и оценки знаний обучающихся с целью улучшения процесса обучения.

Задачами данной разработки являются:
\begin{enumerate}
\item создание модуля для создания новых тестов;
\item создание модуля для создания вопросов;
\item создание модуля для редактирования вопросов;
\item создание модуля для авторизации;
\item создание модуля для регистрации;
\item создание модуля для просмотра результатов пройденных тестов;
\item создание модуля для управления пользователями;
\item создание модуля для пользователя;
\item создание модуля для администратора;
\item реализация системы оценивания по результатам теста;
\item реализация системы хранения тестов и результатов тестов;
\end{enumerate}

\subsection{Требования к данным программной системы}

Входными данными для системы являются:
\begin{itemize}
    \item данные имеющихся тестов;
    \item данные о зарегистрированных пользователях;
    \item данные администратора;
\end{itemize}

Выходными данными для системы являются:
\begin{itemize}
	\item данные создании/изменении тестов;
	\item данные о новых зарегистрированных/добавленных администратором пользователях;
	\item данные о изменении паролей пользователей;
	\item данные результатов пройденных тестов.
\end{itemize}

\subsection{Функциональные требования к программной системе}

В разрабатываемой программно-информационной системе должно
быть предусмотрено наличие два класса пользователей: пользователь и администратор.

Пользователю должны быть доступны следующие функции программы:
\begin{enumerate}
	\item Авторизация.
	\item Регистрация.
	\item Прохождение выбранного теста.
	\item Просмотр собственных результатов пройденных тестов.
\end{enumerate}

Администратору должны быть доступны следующие функции программы:
\begin{enumerate}
	\item Авторизация.
	\item Изменение пароля администратора.
	\item Прохождение выбранного теста.
	\item Просмотр результатов всех пользователей.
	\item Управление пользователями.
	\item Создание новых тестов.
	\item Редактирование имеющихся тестов.
\end{enumerate}

\clearpage

На рисунке ~\ref{user_precedent_diagram:image} в виде диаграммы прецедентов представлены функциональные требования к системе, доступные для пользователя.

\begin{figure}[H]
	\includegraphics[width=1\linewidth]{диаграмма_прецедентов_пользователь}
	\caption{Диаграмма прецедентов пользователя}
	\label{user_precedent_diagram:image}
\end{figure}

\clearpage

На рисунке ~\ref{admin_precedent_diagram:image} представлены дополнительные функциональные требования к системе для администратора.

\begin{figure}[H]
	\includegraphics[width=1\linewidth]{диаграмма_прецедентов_администратор}
	\caption{Диаграмма прецедентов администратора}
	\label{admin_precedent_diagram:image}
\end{figure}

\subsection{Моделирование вариантов использования}

Для разрабатываемой системы тестирования была реализована модель, которая обеспечивает наглядное представление вариантов использования системы.

Она помогает в физической разработке и детальном анализе взаимосвязей объектов.

Диаграмма вариантов описывает функциональное назначение разрабатываемой системы. То есть это то, что система будет непосредственно делать в процессе своего функционирования. Она является исходным концептуальным представлением системы в процессе ее проектирования и разработки. Проектируемая система представляется в виде ряда прецедентов, предоставляемых системой актерам или сущностям, которые взаимодействуют с системой. Актером или действующим лицом является сущность, взаимодействующая с системой извне (например, человек, техническое устройство). Прецедент служит для описания набора действий, которые система предоставляет актеру.

На основании анализа предметной области в программе должны быть реализованы следующие варианты использования:
\begin{enumerate}
\item ВИ "Запустить программу". Данный прецедент позволяет пользователю открыть программу.
\item ВИ "Авторизация". Данный прецедент позволяет пользователю авторизоваться.
\item ВИ "Выбрать желаемый тест". Данный прецедент позволяет пользователю выбрать желаемый тест.
\item ВИ "Пройти выбранный тест". Данный прецедент позволяет пользователю открыть пройти выбранный тест.
\item ВИ "Добавить вопрос". Данный прецедент позволяет пользователю добавить вопрос в выбранном тесте.
\item ВИ "Удалить вопрос". Данный прецедент позволяет пользователю удалить вопрос в выбранном тесте.
\item ВИ "Просмотреть результаты". Данный прецедент позволяет пользователю посмотреть результаты предыдущих тестов.
\item ВИ "Создать новый тест". Данный прецедент позволяет пользователю создать новый тест.
\end{enumerate}

Диаграмма прецедентов представлена на рисунке ~\ref{precedent:image}.

\begin{figure}[ht]
	\includegraphics[width=1\linewidth]{precedent}
	\caption{Диаграмма прецедентов}
	\label{precedent:image}
\end{figure}

\subsection{Описание вариантов использования}

Данные варианта использования "Пройти выбранный тест".

Входными данными является выбранный тест, имя пользователя.

Выходными данными прецедента "Пройти выбранный тест" являются вопросы с ответами и результат прохождения теста.

Основной исполнитель: Пользователь.

Заинтересованные лица и их требования: Пользователь хочет пройти тест.

Предусловие: поля для ввода имени, ответа на вопрос должны быть заполнены.

Постусловие: приложение проверит введены ли данные, если нет, сообщит об этом пользователю.

Основной успешный сценарий:
\begin{enumerate}
	\item Пользователь запускает программу.
	\item Пользователь проходит авторизацию, введя своё имя.
	\item Пользователь попадает в меню.
	\item Пользователь выбирает желаемый тест.
	\item Пользователь нажимает кнопку "Пройти тест".
	\item Пользователь попадает в окно теста.
	\item Пользователь выбирает или вводит ответы.
	\item Пользователь по окончании теста видит его результат.
	\item Пользователь закрывает приложение.
\end{enumerate}

Данные варианта использования "Создать новый тест".

Входными данными является имя пользователя.

Выходными данными прецедента "Создать новый тест" являются новый тест с вопросами и ответами.

Основной исполнитель: Пользователь.

Заинтересованные лица и их требования: Пользователь хочет создать тест и добавить в него вопрос

Предусловие: поля для ввода имени, текста вопроса и его ответа должны быть заполнены.

Постусловие: приложение проверит введены ли данные, если нет, сообщит об этом пользователю.

Основной успешный сценарий:
\begin{enumerate}
	\item Пользователь запускает программу.
	\item Пользователь проходит авторизацию, введя своё имя.
	\item Пользователь попадает в меню.
	\item Пользователь выбирает желаемый тест.
	\item Пользователь нажимает кнопку "Создать тест".
	\item Пользователь попадает в окно создания теста".
	\item Пользователь нажимает кнопку "Создать новый тест".
	\item Пользователь нажимает кнопку "Вернуться в меню".
	\item Пользователь попадает в меню.
	\item Пользователь нажимает кнопку "Добавить вопрос".
	\item Пользователь попадает в меню добавления вопроса.
	\item Пользователь нажимает кнопку добавить базовый вопрос.
	\item Пользователь в окне создания базового вопроса вводит текст и ответ вопроса.
	\item Пользователь нажимает кнопку добавить вопрос.
	\item Пользователь закрывает приложение.
\end{enumerate}

\subsection{Требования к оформлению документации}

Разработка программной документации и программного изделия должна производиться согласно ГОСТ 19.102-77 и ГОСТ 34.601-90. Единая система программной документации.
