\addcontentsline{toc}{section}{СПИСОК ИСПОЛЬЗОВАННЫХ ИСТОЧНИКОВ}

\begin{thebibliography}{9}

    \bibitem{python} Лутц М. Изучаем Python, 4-е издание. – Пер. с англ. – СПб.: Символ-Плюс, 2011. – 1280 с.
    \bibitem{python} Златопольский Д.М. Основы программирования на языке Python. – М.: ДМК Пресс, 2017. – 284 с.
    \bibitem{python} Лутц М. Программирование на Python, том I, 4-е издание. – Пер. с англ. – СПб.: Символ-Плюс, 2011. – 992 с.
    \bibitem{python} Лутц М. Программирование на Python, том II, 4-е издание. – Пер. с англ. – СПб.: Символ-Плюс, 2011. – 992 с.
	\bibitem{python} Гэддис Т. Начинаем программировать на Python.  – 4-е изд.: Пер. с англ. – СПб.: БХВ-Петербург, 2019. – 768 с.
	\bibitem{python} Лучано Рамальо Python. К вершинам мастерства. – М.: ДМК Пресс, 2016. – 768 с.
	\bibitem{python} Свейгарт, Эл. Автоматизация рутиных задач с помощью Python: практическое руководство для начинающих. Пер. с англ. — М.: Вильямc, 2016. – 592 с.
	\bibitem{python} Рейтц К., Шлюссер Т. Автостопом по Python. – СПб.: Питер, 2017. – 336 с.: ил. – (Серия «Бестселлеры O’Reilly»).
	\bibitem{python} Любанович Билл Простой Python. Современный стиль программирования. – СПб.: Питер, 2016. – 480 с.
	\bibitem{python} Доунсон М. Программируем на Python. - СПб.: Питер, 2014.- 416 с.
	\bibitem{python} Элиенс А. Принципы объектно-ориентированной разработки программ. – М.: Вильямс, 2002. – 496 с.
	\bibitem{python} Грэхем И. Объектно-ориентированные методы: Принципы и практика: пер. с англ. Изд. 3-е. – М: Вильямс, 2004. – 880 с.
	\bibitem{python} Иванова Г.С., Ничушкина Т.Н., Пугачев Е.К. Объектно-ориентированное программирование: Учеб. для вузов, Под. Ред. Г.С. Ивановой. – М.: Изд-во МГТУ им. Н.Э.Баумана, 2001. – 320 с.
	\bibitem{python} Иванова Г.С. Технология программирования: учебник для вузов. – М.: Изд-во МГТУ им. Н.Э. Баумана, 2003. – 320 с.
	\bibitem{python} Прохоренок Н.А. Python 3 и PyQt. Разработка приложений. – СПб.: БХВ-Петербург, 2012. – 704 с.
	\bibitem{python} Россум Г. , Ф.Л.Дж. Дрейк, Д.С. Откидач, [и др.]. Язык программирования Python, 2001 — 454 c.
	\bibitem{GUI} Букунов С. В. Разработка приложений с графическим пользовательским интерфейсом на языке Python : учебное пособие для СПО, С. В. Букунов, О. В. Букунова. — Санкт-Петербург : Лань, 2023. — 88 с.
	\bibitem{GUI} Машнин Т. Создание настольных Python приложений с графическим интерфейсом пользователя, Машин Т. – Москва: ЛитРес: Самиздат, 2021. – 131 с.
	\bibitem{patterns} Гринспан Д. Принципы объектно-ориентированного проектирования. Паттерны проектирования, Д. Гринспан, М. Хайнрих. — Санкт-Петербург : Символ-Плюс, 2017. — 208 с.
	\bibitem{programming} Макконнелл С. Совершенный код, С. Макконнелл. — Санкт-Петербург : Питер, 2018. — 896 с.
\end{thebibliography}
