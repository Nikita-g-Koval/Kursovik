\addcontentsline{toc}{section}{СПИСОК ИСПОЛЬЗОВАННЫХ ИСТОЧНИКОВ}

\begin{thebibliography}{9}

    \bibitem{python} Лутц М. Изучаем Python, 4-е издание. – Пер. с англ. – СПб.: Символ-Плюс, 2011. – 1280 с.
    \bibitem{python} Златопольский Д.М. Основы программирования на языке Python. – М.: ДМК Пресс, 2017. – 284 с.
    \bibitem{python} Лутц М. Программирование на Python, том I, 4-е издание. – Пер. с англ. – СПб.: Символ-Плюс, 2011. – 992 с.
    \bibitem{python} Лутц М. Программирование на Python, том II, 4-е издание. – Пер. с англ. – СПб.: Символ-Плюс, 2011. – 992 с.
	\bibitem{python} Гэддис Т. Начинаем программировать на Python.  – 4-е изд.: Пер. с англ. – СПб.: БХВ-Петербург, 2019. – 768 с.
	\bibitem{python} Лучано Рамальо Python. К вершинам мастерства. – М.: ДМК Пресс, 2016. – 768 с.
	\bibitem{python} Свейгарт, Эл. Автоматизация рутиных задач с помощью Python: практическое руководство для начинающих. Пер. с англ. — М.: Вильямc, 2016. – 592 с.
	\bibitem{python} Рейтц К., Шлюссер Т. Автостопом по Python. – СПб.: Питер, 2017. – 336 с.: ил. – (Серия «Бестселлеры O’Reilly»).
	\bibitem{python} Любанович Билл Простой Python. Современный стиль программирования. – СПб.: Питер, 2016. – 480 с.: – (Серия «Бестсепперы O’Reilly»).  
\end{thebibliography}
