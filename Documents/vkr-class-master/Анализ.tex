% !TeX spellcheck = ru_RU-Russian
\section{Анализ предметной области}
\subsection{Системы тестирования и оценки знаний, основные цели и задачи}

Системы тестирования знаний представляют собой программные комплексы, предназначенные для автоматизированной оценки уровня знаний, навыков и компетенций обучающихся в различных областях. Они являются важным инструментом в образовательном процессе, позволяющим преподавателям эффективно контролировать усвоение материала, а студентам - оценивать свои знания и выявлять пробелы.

Основные цели и задачи систем тестирования знаний:
\begin{itemize}
	\item автоматизация оценки: снижение трудовых затрат на проверку работ, особенно при большом количестве обучающихся;
	\item объективность оценки: минимизация субъективного фактора при оценке знаний, основанной на четких критериях и алгоритмах;
	\item повышение эффективности обучения: предоставление оперативной обратной связи студентам о результатах тестирования, что позволяет им своевременно корректировать свою учебную деятельность;
	\item дифференцированный подход: возможность создания тестов различной сложности, учитывающих индивидуальные особенности обучающихся;
	\item разнообразие форматов: поддержка различных типов тестовых заданий (множественный выбор, открытые вопросы, соответствие, перестановка и т.д.), что позволяет оценивать знания с разных сторон;
	\item контроль за ходом обучения: мониторинг успеваемости обучающихся, выявление отстающих и оказание им своевременной помощи;
	\item управление знаниями: создание и поддержание базы тестовых заданий, что позволяет повторно использовать их в различных учебных курсах;
	\item статистический анализ: предоставление данных об уровне усвоения материала по различным темам, что позволяет преподавателям корректировать учебные планы и методы обучения;
	\item удобство использования: интуитивно понятный интерфейс для преподавателей и студентов, облегчающий процесс создания, проведения и прохождения тестирования;
	\item безопасность: обеспечение конфиденциальности результатов тестирования и защита от несанкционированного доступа к системе.
\end{itemize}

\subsection{Классификация систем тестирования знаний}

В настоящее время определено, что автоматизированные цифровые тесты помогают
получить более объективные оценки уровня знаний, умений, навыков,
проверить соответствие требований к подготовке выпускников вузов заданным
стандартам, выявить пробелы в подготовке персонала.

Тест – это инструмент, состоящий из квалиметрически выверенной системы тестовых заданий, стандартизированной процедуры проведения и заранее спроектированной технологии обработки и анализа результатов, предназначенной для измерения качеств и свойств личности, изменение которых возможно в процессе систематического обучения. Или тест – краткое стандартизированное испытание, допускающее количественную оценку результатов на основе их
статистической обработки. Отбор структура заданий теста зависит от того, какие показатели и факторы интересуют исследователя данной группы лиц. Каждое из заданий теста по своей сути представляет для испытуемого вопрос, проблему. Ответ на вопрос – это всегда устранение некоторых сомнений, колебаний, неопределенности в рассматриваемой ситуации с целью получения новых, более точных знаний.

С точки зрения целей применения можно выделить:
\begin{itemize}
	\item тесты достижений;
	\item критериально-ориентированные тесты, позволяющие сопоставить
	уровень индивидуальных учебных достижений с полным объемом знаний,
	умений и навыков;
	\item нормативно-ориентированные тесты, сравнивающие испытуемых по
	уровню их учебных достижений;
	\item аттестационные тесты, определяющие степень обученности;
	\item тесты прогнозирования результатов обучения по выбранной
	технологии обучения.
\end{itemize}

\subsubsection{Основные компоненты системы тестирования знаний}

В системах тестирования знаний можно выделить несколько ключевых компонентов, каждый из которых выполняет определенную функцию и обеспечивает работоспособность системы в целом.

\paragraph{Модуль управления пользователями}
	
Отвечает за управление всеми пользователями системы, включая преподавателей, студентов, администраторов и других заинтересованных лиц. 

Характеризуется следующими функциями:
\begin{itemize}
	\item регистрация пользователей: предоставление возможности создания новых учетных записей с указанием необходимой информации;
	\item авторизация пользователей: проверка логина и пароля при входе в систему, предоставление доступа к функциональности в соответствии с ролью пользователя;
	\item просмотр и редактирование профилей пользователей: предоставление возможности просмотра и изменения личной информации пользователей;
	\item удаление учетных записей;
	\item права доступа: пользователь может только проходить тесты и просматривать свои результаты, администратор имеет право создавать и редактировать тесты.
\end{itemize}

\paragraph{Модуль создания и редактирования тестов}
	
Предназначен для разработки и модификации тестов с использованием различных типов вопросов и настроек. 
	
Функции: 
\begin{itemize}
	\item создание новых тестов: предоставление интерфейса для создания новых тестов с указанием названия и других параметров;
	\item добавление вопросов: Возможность добавления вопросов различных типов (ввод ответа, единственный выбор, множественный выбор ответов);
	\item предпросмотр тестов: возможность предварительного просмотра теста с точки зрения студента для проверки корректности отображения вопросов и ответов;
	\item импорт и экспорт тестов: поддержка импорта тестов из других систем или файлов (например, в формате JSON) и экспорта тестов в различные форматы.
\end{itemize}

Технические аспекты: 
\begin{itemize}
	\item использование визуального редактора для создания вопросов;
	\item поддержка различных типов данных для хранения вопросов и ответов;
	\item реализация валидации данных для проверки корректности введенной информации.
\end{itemize}

	
\paragraph{Модуль проведения тестирования}
	
Обеспечивает возможность проходить тесты и получать результаты.

Функции: 
\begin{itemize}
	\item доступ к тестам: предоставление списка доступных тестов для студента;
	\item начало и завершение тестирования: возможность начала и завершения тестирования;
	\item отображение вопросов: отображение вопросов в удобном формате с возможностью выбора ответов или ввода текста;
	\item сохранение ответов: автоматическое сохранение ответов студента в процессе тестирования;
	\item предоставление результатов: отображение результатов тестирования после его завершения, включая количество правильных и неправильных ответов, итоговый балл и оценку;
	\item поддержка различных типов тестирования - возможность настройки типов вопросов: вопрос с самостоятельным вводом ответа; вопрос с единственным выбором ответа, вопрос с множественным выбором ответов.
\end{itemize}

\paragraph{Модуль автоматической проверки}

Автоматически проверяет ответы на вопросы, требующие однозначного ответа, и выставляет предварительные результаты.

Функции: 
\begin{itemize}
	\item сравнение ответов студента с правильными ответами;
	\item вычисление баллов: вычисление баллов за правильные ответы;
	\item формирование отчета о предварительных результатах: создание отчета о предварительных результатах тестирования с указанием количества правильных и неправильных ответов, итогового балла и оценки.
\end{itemize}	

\subsection{Сравнение систем тестирования знаний}
Автоматизированные системы контроля знаний являются эффективным и востребованным инструментом в современном образовательном процессе, однако требуют дальнейшего развития с учетом современных требований к образовательным технологиям и необходимости обеспечения объективности оценивания.
Для улучшения качества проектируемого продукта, следует провести анализ уже существующих решений в этой сфере. Наиболее популярные системы для оценки и контроля знаний: Easy Quizzy \cite{EasyQuizzy}, My TestX \cite{MyTestX} и INDIGO\cite{INDIGO}. Приведем сравнение систем в таблице \ref{systems:table}

\begin{xltabular}{\textwidth}{|>{\raggedright\arraybackslash}p{5.5cm}|>{\centering\arraybackslash}X|>{\centering\arraybackslash}X|>{\centering\arraybackslash}X|>{\centering\arraybackslash}X|} 
	\caption{Сравнение функционала систем Easy Quizzy, My TestX, INDIGO и проектируемой\label{systems:table}} \\ \hline
	Параметр & Easy Quizzy & My TestX & INDIGO & Проекти-руемая \\ \hline
	\endfirsthead
	\continuecaption{Продолжение таблицы \ref{systems:table}}
	Параметры & Easy Quizzy & My TestX & INDIGO & Проекти-руемая \\ \hline 
	\endhead
	Управление пользователями & - & - & + & +  \\ \hline
	Хранение данных о пользователях & + & + & + & +  \\ \hline
	Создание тестовых заданий различных типов & - & - & + & +  \\ \hline
	Минимальные системные требования  & - & - & + & +  \\ \hline
	Защита тестов от подсматривания ответов & + & + & + & +  \\ \hline
	Понятный интерфейс & - & + & - & +  \\ \hline
	Поддержка Windows & + & + & + & + \\ \hline
	Клиентская база & + & + & + & -  \\ \hline
	Составление диаграммы результатов & - & - & - & +   \\ \hline
	Простота использования & - & + & - & +  \\ \hline
	Возможность копирования тестовых наборов & - & - & + & + \\ \hline
	Возможность сохранения графиков результатов пользователей & - & - & + & + \\ \hline  
\end{xltabular}
