\section{Анализ предметной области}
\subsection{Характеристика предприятия и его деятельности. Тест длинного заголовка, не должен содержать переносы}

Системы управления тестированием используются для хранения информации о том, как должным образом проводить тестирование, осуществление очередности проведения тестирования в соответствии с его планом, а также для получения информации в виде отчетов о стадии тестирования и качестве тестируемого продукта. Инструменты имеют различные подходы к тестированию и, таким образом, включают в себя различные наборы функций. Обычно они используются для планирования ручного тестирования, сбора данных о результатах прохождения чек-листов и тест-кейсов, а также для получения оперативной информации в виде отчетов. Системы управления тестированием помогают оптимизировать процесс тестирования и обеспечивают быстрый доступ к анализу данных, средствам совместной работы и более качественному взаимодействию между несколькими проектными группами. Многие системы управления тестированием включают в себя возможность работы с требованиями.

Инструменты управления тестированием дают командам возможность консолидировать и структурировать процесс тестирования с помощью одной из систем управления тестированием вместо установки нескольких приложений, которые предназначены для управления только одним процессом или его частью. Некоторые приложения включают передовые инструментальные панели для тщательного отслеживания ключевых показателей, что позволяет легко получать необходимую информацию о стадиях процесса тестирования и качестве тестируемого продукта.

После старта тестирования проекта члены команды могут взаимодействовать через одну из систем управления тестирования путём создания тест-кейсов, чек-листов, назначая ответственных за прохождение их лиц, что упрощает и улучшает качество взаимодействия лиц, проводящих тестирование в рамках конкретного проекта. При создании или прохождении тестов и чек-листов пользователи могут получить доступ к различным функциям систем управления тестированием, которые автоматизируют данную деятельность и благотворно влияют на скорость и качество её выполнения.

\subsection{Система тестирования, их классификация}

В настоящее время определено, что педагогические тесты помогают
получить более объективные оценки уровня знаний, умений, навыков,
проверить соответствие требований к подготовке выпускников вузов заданным
стандартам, выявить пробелы в подготовке персонала.

Тест – это инструмент, состоящий из квалиметрически выверенной системы тестовых заданий, стандартизированной процедуры проведения и заранее спроектированной технологии обработки и анализа результатов, предназначенной для измерения качеств и свойств личности, изменение которых возможно в процессе систематического обучения. Или тест – краткое стандартизированное испытание, допускающее количественную оценку результатов на основе их
статистической обработки. Отбор структура заданий теста зависит от того, какие показатели и факторы интересуют исследователя данной группы лиц. Каждое из заданий теста по своей сути представляет для испытуемого вопрос, проблему. Ответ на вопрос – это всегда устранение некоторых сомнений, колебаний, неопределенности в рассматриваемой ситуации с целью получения новых, более точных знаний.

С точки зрения целей применения можно выделить:

тесты достижений;

критериально-ориентированные тесты, позволяющие сопоставить
уровень индивидуальных учебных достижений с полным объемом знаний,
умений и навыков;

нормативно-ориентированные тесты, сравнивающие испытуемых по
уровню их учебных достижений;

аттестационные тесты, определяющие степень обученности;

тесты прогнозирования результатов обучения по выбранной
технологии обучения.