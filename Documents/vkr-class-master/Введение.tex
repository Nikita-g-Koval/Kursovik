\section*{ВВЕДЕНИЕ}
\addcontentsline{toc}{section}{ВВЕДЕНИЕ}

Системы управления тестированием используются для хранения информации о том, как должным образом проводить тестирование, осуществление очередности проведения тестирования в соответствии с его планом, а также для получения информации в виде отчетов о стадии тестирования и качестве тестируемого продукта. Инструменты имеют различные подходы к тестированию и, таким образом, включают в себя различные наборы функций. Обычно они используются для планирования ручного тестирования, сбора данных о результатах прохождения чек-листов и тест-кейсов, а также для получения оперативной информации в виде отчетов. Системы управления тестированием помогают оптимизировать процесс тестирования и обеспечивают быстрый доступ к анализу данных, средствам совместной работы и более качественному взаимодействию между несколькими проектными группами. Многие системы управления тестированием включают в себя возможность работы с требованиями.

Инструменты управления тестированием дают командам возможность консолидировать и структурировать процесс тестирования с помощью одной из систем управления тестированием вместо установки нескольких приложений, которые предназначены для управления только одним процессом или его частью. Некоторые приложения включают передовые инструментальные панели для тщательного отслеживания ключевых показателей, что позволяет легко получать необходимую информацию о стадиях процесса тестирования и качестве тестируемого продукта.

После старта тестирования проекта члены команды могут взаимодействовать через одну из систем управления тестирования путём создания тест-кейсов, чек-листов, назначая ответственных за прохождение их лиц, что упрощает и улучшает качество взаимодействия лиц, проводящих тестирование в рамках конкретного проекта. При создании или прохождении тестов и чек-листов пользователи могут получить доступ к различным функциям систем управления тестированием, которые автоматизируют данную деятельность и благотворно влияют на скорость и качество её выполнения.

\emph{Цель настоящей работы} – создание наборов вопросов для тестирования пользователей, сбор результатов тестирования пользователей путём создание системы тестирования. Для достижения поставленной цели необходимо решить \emph{следующие задачи:}
\begin{itemize}
\item провести анализ предметной области;
\item разработать концептуальную модель системы тестирования;
\item спроектировать систему тестирования;
\item реализовать систему тестирования средствами графических библиотек.
\end{itemize}

\emph{Структура и объем работы.} Отчет состоит из введения, 4 разделов основной части, заключения, списка использованных источников, 2 приложений. Текст выпускной квалификационной работы равен \formbytotal{page}{страниц}{е}{ам}{ам}.

\emph{Во введении} сформулирована цель работы, поставлены задачи разработки, описана структура работы, приведено краткое содержание каждого из разделов.

\emph{В первом разделе} на стадии описания технической характеристики предметной области приводится сбор информации о задачах, для которых осуществляется разработка системы тестирования.

\emph{Во втором разделе} на стадии технического задания приводятся требования к системе тестирования.

\emph{В третьем разделе} на стадии технического проектирования представлены проектные решения для системы тестирования.

\emph{В четвертом разделе} приводится список классов и их методов, использованных при разработке системы тестирования, производится тестирование разработанного приложения.

В заключении излагаются основные результаты работы, полученные в ходе разработки.

В приложении А представлен графический материал.
В приложении Б представлены фрагменты исходного кода. 
