% !TeX spellcheck = ru_RU-Russian
\section*{ВВЕДЕНИЕ}
\addcontentsline{toc}{section}{ВВЕДЕНИЕ}

В современном образовательном процессе, где акцент делается на индивидуальном подходе и непрерывном обучении, системы для оценки и контроля знаний играют ключевую роль. Они представляют собой мощный инструмент для автоматизации оценки уровня знаний, умений и навыков обучающихся, предоставляя объективную и оперативную обратную связь.

Эффективная система тестирования знаний заменяет традиционные методы контроля и трансформирует сам процесс обучения, позволяя преподавателям оперативно выявлять пробелы в знаниях, корректировать учебные планы и предлагать индивидуальные траектории обучения. Студенты, в свою очередь, получают возможность оценивать свой прогресс, выявлять свои сильные и слабые стороны и фокусироваться на областях, требующих дополнительного внимания.

Система тестирования уровня знаний предназначена для оценки знаний и навыков учащихся, студентов или сотрудников в определенной области. Она позволяет оценить уровень подготовленности людей и выявить их сильные и слабые стороны. Такая система может не только проверить знания, но и определить необходимость дальнейшего обучения или повышения квалификации, а также помогает контролировать уровень достижений и прогресса в обучении.

\emph{Цель настоящей работы} – разработка программно-информационной системы, позволяющей создавать тесты, осуществлять тестирование пользователей, осуществлять сбор и анализ результатов тестирования. Для достижения поставленной цели необходимо решить \emph{следующие задачи:}
\begin{itemize}
\item провести анализ предметной области;
\item разработать концептуальную модель программно-информационной системы для оценки и контроля знаний;
\item спроектировать программно-информационную систему для оценки и контроля знаний;
\item реализовать программно-информационную систему для оценки и контроля знаний средствами графических библиотек.
\end{itemize}

\emph{Структура и объем работы.} Отчет состоит из введения, 4 разделов основной части, заключения, списка использованных источников, 2 приложений. Текст выпускной квалификационной работы равен \formbytotal{lastpage}{страниц}{е}{ам}{ам}.

\emph{Во введении} сформулирована цель работы, поставлены задачи разработки, описана структура работы, приведено краткое содержание каждого из разделов.

\emph{В первом разделе} на стадии описания технической характеристики предметной области приводится информация о целях и задачах систем тестирования знаний, а также её классификация и основные компоненты.

\emph{Во втором разделе} на стадии технического задания приводятся требования к разрабатываемому приложению.

\emph{В третьем разделе} на стадии технического проектирования представлены проектные решения для приложения.

\emph{В четвертом разделе} приводится список классов и их методов, использованных при разработке приложения, производится тестирование разработанного приложения.

В заключении излагаются основные результаты работы, полученные в ходе разработки.

В приложении А представлен графический материал.
В приложении Б представлены фрагменты исходного кода. 
