% !TeX spellcheck = ru_RU-Russian
\abstract{РЕФЕРАТ}

Объем работы равен \formbytotal{lastpage}{страниц}{е}{ам}{ам}. Работа содержит \formbytotal{figurecnt}{иллюстраци}{ю}{и}{й}, \formbytotal{tablecnt}{таблиц}{у}{ы}{}, \arabic{bibcount} библиографических источников и \formbytotal{числоПлакатов}{лист}{}{а}{ов} графического материала. Количество приложений – 2. Графический материал представлен в приложении А. Фрагменты исходного кода представлены в приложении Б.

Перечень ключевых слов: система, тестирование, вопрос, ответ, результат, пользователь, администратор, окно, класс, модуль, компонент, диаграмма, автоматизация, Tkinter, JSON.

Объектом разработки является создание программно-информационной системы для оценки и контроля, позволяющей создавать тесты, осуществлять тестирование пользователей, осуществлять сбор и анализ результатов тестирования.

Целью выпускной квалификационной работы является автоматизация процесса тестирования знаний обучающихся.

В процессе создания программно-информационной системы были выделены основные компоненты путем создания информационных блоков, использованы классы и методы модулей, обеспечивающие работу с предметной области, а также корректную работу приложения, разработаны следующие модули: модуль теста, модуль просмотра результатов, модуль создания теста, модуль добавления вопросов, модуль просмотра вопросов теста, модуль управления пользователями.

При разработке приложения использовалась графическая библиотека "<Tkinter">.

\selectlanguage{english}
\abstract{ABSTRACT}
  
The volume of work is \formbytotal{lastpage}{page}{}{s}{s}. The work contains \formbytotal{figurecnt}{illustration}{}{s}{s}, \formbytotal{tablecnt}{table}{}{s}{s}, \arabic{bibcount} bibliographic sources and \formbytotal{числоПлакатов}{sheet}{}{s}{s} of graphic material. The number of applications is 2. The graphic material is presented in annex A. The layout of the site, including the connection of components, is presented in annex B.

The list of keywords: system, testing, question, answer, result, user, administrator, window, class, module, component, diagram, automation, Tkinter, JSON.

The object of the development is to create a software and information system for evaluation and control, which allows you to create tests, test users, collect and analyze test results.

The purpose of the final qualification is to automate the process of testing students' knowledge.

In the process of creating a software and information system, the main components were identified by creating information blocks, classes and module methods were used to work with the subject area, as well as the correct operation of the application, the following modules were developed: test module, results viewing module, test creation module, question addition module, test question viewing module, management module by users.

The graphic library <<Tkinter>> was used during the development of the application.
\selectlanguage{russian}
