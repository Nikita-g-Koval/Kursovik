\section*{ЗАКЛЮЧЕНИЕ}
\addcontentsline{toc}{section}{ЗАКЛЮЧЕНИЕ}

В настоящее время определено, что педагогические тесты помогают получить более объективные оценки уровня знаний, умений, навыков, проверить соответствие требований к подготовке выпускников вузов заданным стандартам, выявить пробелы в подготовке студентов.
  
Система тестирования – это система, обладающая двумя главными системными факторами: содержательным составом тестовых заданий, образующих наилучшую целостность, и нарастанием трудности от задания к заданию. Принцип нарастания трудности и позволяет определить уровень знаний и умений по контролируемой дисциплине, а обязательное ограничение времени тестирования – выявить наличие навыков и умений.

Основные результаты работы:

\begin{enumerate}
\item Проведен анализ предметной области. Выявлена необходимость использовать систему тестирования.
\item Разработана концептуальная модель система тестирования. Разработана модель данных системы. Определены требования к системе.
\item Осуществлено проектирование системы тестирования. Разработана архитектура клиентской части. Разработан пользовательский интерфейс системы тестирования.
\item Реализована и протестирована система тестирования. Проведено модульное и системное тестирование.
\end{enumerate}

Все требования, объявленные в техническом задании, были полностью реализованы, все задачи, поставленные в начале разработки проекта, были также решены.

Готовый рабочий проект представлен оконным приложением.  
