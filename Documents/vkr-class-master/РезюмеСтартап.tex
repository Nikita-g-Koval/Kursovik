% !TeX spellcheck = ru_RU-Russian
\section*{РЕЗЮМЕ СТАРТАП-ПРОЕКТА}
\addcontentsline{toc}{section}{РЕЗЮМЕ СТАРТАП-ПРОЕКТА}
Название: «\Тема\ \ТемаВтораяСтрока». Проект представляет собой систему, предназначенную  для автоматизации тестирования знаний обучающихся. Данная система оптимизирует следующие задачи: регистрация и авторизация пользователей; тестирование знаний пользователей; автоматическая проверка ответов на вопросы теста; создание и редактирование тестов; управление пользователями; просмотр результатов и создание диаграммы для их визуализации; два модуля приложения, для пользователя и для администратора. Были проанализированы современные аналоги и выявлены их \textit{отрицательные стороны}:
\begin{enumerate}
	\item Сложность и неудобство использования. Большинство программно-информационных систем для оценки и контроля знаний имеют неудобный, неоптимизированный интерфейс, который перегружен большим количеством функций. Данный недостаток значительно усложняет как процесс создания тестов, так и их прохождение. Так же плохой интерфейс вызывает раздражение и тем самым вредит объективности оценки знаний.
	\item Высокая стоимость: самые распространённые системы тестирования знаний для доступа ко всем функциям требуют оформления подписки или покупку полной лицензионной версии, что требует значительных денежных затрат.
	\item Многие программно-информационные системы в данной области являются устаревшими: не поддерживают современные технологии хранения тестов, в следствие чего их редактирование требует много времени.
	\item В некоторых системах отсутствует возможность визуализации результатов или же их просмотр является временно-затратным процессом.
	\item Разнообразие типов вопросов ограничено и чаще всего включает только самостоятельный ввод ответа, что создаёт однообразие и приводит к быстрой утомляемости.
	\item Отсутствие управления пользователями, из-за чего администратор не может добавлять, удалять и редактировать пользователей.
\end{enumerate} 


Созданная программно-информационная система позволит избежать вышеперечисленных трудностей без негативного влияния на процесс работы и скорость выполнения поставленных задач.

\textit{Актуальность} разрабатываемой системы связана с потребностью учебных заведений в автоматизации оценки и контроля знаний, оптимизации создания и редактирования тестов, управлении пользователями и удобном просмотре результатов.

\textit{Целью} является разработка программно-информационной системы для оценки и контроля знаний. 


\textit{Отличительные особенности} программно-информационной системы по сравнению с уже существующими решениями и программами:
\begin{enumerate}
	\item Простой и интуитивно понятный графический интерфейс с двумя модулями: для пользователя и для администратора.
	\item Возможность создания тестов и вопросов с удобным хранением в файле формата JSON.
	\item Поддержка различных типов вопросов: с вводом ответа, с выбором единственного ответа и с множественным выбором.
	\item Удобный просмотр результатов прохождения тестов, а также возможность визуализации лучших попыток с помощью диаграммы.
	\item Управление пользователями: администратор может добавлять, удалять и редактировать пользователей.
\end{enumerate}


В разработанной программно-информационной системе для оценки и контроля знаний реализованы такие задачи как: регистрация и авторизация пользователей, тестирование знаний пользователей; автоматическая проверка ответов на вопросы теста; создание и редактирование тестов; управление пользователями; просмотр результатов и создание диаграммы для их визуализации; два модуля приложения, для пользователя и для администратора.

Идея создание проекта появилась в сентябре 2023 года, разработка была начата в декабре 2023 года. А с марта 2025 производилось тестирование системы путем создания новых тестовых наборов и их прохождения. Проект планируется закончить к началу 2026 года. Также будет осуществлена регистрация программно-информационной системы для избежания плагиата и обеспечения исключительности прав.
\newpage
\textit{Бизнес-цели:}
 \begin{enumerate}
 	\item Добавить возможность создания комплексных тестов, которые будут включать в себя вопросы по разным темам и по завершении теста выдавать общую характеристику пользователя.
 	\item Добавить функцию анализа результатов, и в зависимости от выявленных сильных и слабых сторон рекомендовать соответствующее обучение.
 	\item Создать мобильную версию приложения.
 	\item Внедрить готовую систему в процесс обучения студентов.
 \end{enumerate}


